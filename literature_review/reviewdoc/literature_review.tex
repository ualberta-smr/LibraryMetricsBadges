\documentclass[12pt]{article}

    \usepackage[hyphens,spaces,obeyspaces]{url}
    

\title{CMPUT 499 Literature Review}
\author{Monica Bui}

\begin{document}

\begin{titlepage}
    \centering
    \large
    \vspace{1cm}
    CMPUT 499: Mining Sofware Repositories \\
    \vspace{1cm}
    Literature Review \\
    \vspace{1cm}
    Monica Bui
\end{titlepage}

% TODO add bibliography 

% Literature review discusses published inforamtion in subject area
% Focus on summarizing source and introducing my own understandings of topic

\newpage
\section{Introduction}
% Give idea of the topic of the literature review, patterns + theme
% Work in Progress. This will be edited a lot.
Third-party software libraries and Application Programming Interfaces (APIs) offer a way for developers
to use existing features and funuctionalities to build their projects without having to re-invent the wheel.
There are many libraries to use for different programming languages whether it is open source 
or hosted elsewhere. Although having this much freedom for deciding on what libraries to use is flexible,
it is also conflicting to figure out which one is best for your project. It is difficult to choose because
there are a variety of factors involved such as functionality and developer support. 
In this review, we look at several diferent articles to help analyze various properties of libraries and 
to determine a helpful comparison between them.

\section{Article Review}
% Not sure how to format articles. Could start with one article per paragraph.

% That would take to long. Format by theme? 
% Recommenders, Badges, API Comparisons by Metrics, Usage, etc.

\subsection{Library Recommenders}
\paragraph{}
% TODO make paragraph shorter. Just needed to jot ideas down before doing that
One problem is researching new analogical libraries to use for different programming languages that has similar functionalities 
to the ones you currently know. Chen's et. al \cite{analogical} paper discusses their library recommender 
that compiles a list of libraries from community resources such as blogs and Q\&A sites like StackOverflow \cite{stackoverflow} 
and outputs a list of recommended libraries in the developer's language of choice. 
This is implemented by first mining tags on questions posted online on StackOverflow \cite{stackoverflow}.
These tags are split into two knowledge bases: relational and categorical. Respectively, relational knowledge is how pairs of
tags are correlated to each other eg. Java and JUnit while categorical knowledge consists of how tags are grouped into categories such as 
language, operating system, concept, or library with both bases being analyzed by NLP. 
The key idea here is that having different seperated tag categories and relatonships between tags often mentioned together
allowed for a simpler way to recommend a new library. With the database in place, users can then search for recomendations through the built
web application called SimilarTech \cite{similartech}.
While trying out the application myself, I found that search results would only yield for 
libraries that are mentioned in specific contexts.
Axios \cite{axios} a popular Javascript HTTP library should output expected 
recomendations like Requests \cite{requests} module for Python but the actual list printed was empty.
While the number of languages it can suggest libraries for is limited to 5, the precision metric is impressive
with 1 language at 81\% and with 5 being at 67\% showing its potential to grow in the future.

\paragraph{}
Going back to the key problem of looking for the right library to use, Uddin's et. al \cite{opiner} article
highlights their approach to this by looking at personal developer opinions's on different resources
and how it affects the reader's decision. Furthermore, the sentiment behind this can be used to indicate if its 
a positive, questionable, or negative source to use.


\subsection{Github Badges}
\paragraph{}
TODO) Talk about NPM repo badge article


\subsection{Metrics}
\paragraph{}
TODO) Talk about Fernando's article

\newpage 
\section{Conclusion}
% Summarize key points


\newpage
\bibliographystyle{IEEEtran}
\bibliography{literature_review}
\end{document}