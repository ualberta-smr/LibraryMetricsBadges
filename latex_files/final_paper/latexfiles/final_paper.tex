\documentclass[12pt, letterpaper]{article}

\usepackage[hyphens,spaces,obeyspaces]{url}
\usepackage{graphicx}
\graphicspath{ {./images/} }
    

\title{CMPUT 499 Final Paper}
\author{Monica Bui}

\begin{document}

\begin{titlepage}
    \centering
    \large
    \vspace{1cm}
    Exploring Software Library Metrics \\ 
    with Repository Badges \\
    \vspace{1cm}
    by \\
    \vspace{1cm}
    Monica Bui \\
    University of Alberta \\
    bui1@ualberta.ca \\
\end{titlepage}

\tableofcontents

\listoffigures
\newpage

\section{Introduction}
% How do developers select libraries
% Iinstead of using websites, and online sources
% we look to create badges that can be shown on the spot to users
% to help them seee metrics and pick the best library

Libraries, Frameworks, and Application Programming Interfaces (APIs) provide developers a way to reuse existing functionalities
built by someone else without having to re-implement already built features. Given a large collection of libraries out there,
it is often difficult and not clear how to select the best one to use for your own project. 


Developers may resort to first doing a general search of their desired library features with search results indicating
various resources such as a Q\&A website like StackOverflow \cite{stackoverflow} or a website that hosts open source libraries such as Github \cite{github}.


Previous research has examined different, inner aspects of libraries that may have mentioned in the above
online resources \cite{githubbadges,apiwave,metrics,opinerarticle,analogical}. These aspects or defined 
% dont explicity reccomend libraries. Allow develoeprs to make their
% own judgement to select librarie. Another quanittiative tool
% to help compare b/w 

\section{Related Work}
\subsection{Library Selection Methods}
% This work focused on explicity recomending libraries versus. 
% helping users pick which ones to use. My work is related to X while it 
% differentiates itself from Y

% Library COmparison WEbsite (same method to solve similar problem)
% Library Recommender (Opiner differe methods but similar problem)
% Analogical Libraries (differnt method , same problem)
\subsection{Metrics}
\subsubsection{Quality Assurance}
% Security
% Build
% Test code coverage

\subsubsection{Community Support}
% Stack Overflow

\subsubsection{Repository General Information}
% Pull Requests
% Releases
% Issue Stats
% Commits 

\subsection{Summary}

\section{Background}

\subsection{Badges}
% Trockman's work
% Look at what make's a good badge, not looking at how it can help developers choose libaries
\subsection{Online Badge Services}
% Shields io

\section{Badge Implementation}
\subsection{Overarching Structure of Scripts}
% Insert diagram of how we made our badges

\subsection{Security}
\subsubsection{Definition}
The security badge is inspired by Mora's et. al \cite{metrics} security metric implementation.
However, there are limitations related to classifying some security vulnerabilities due to inaccurate
issue descriptions. These inaccuracies would suggest that there was a security problem but in reality was another issue altogether.
To avoid this conflict brought by issue descriptions, 
we propose to use another existing tool called SpotBugs \cite{spotbugs}.
From the SpotBugs \cite{spotbugs} website description itself, the program \textit{uses static analysis to look for bugs in Java code}.
In conjunction with the FindSecBugs \cite{findsecbugs} plugin to provide a larger data set of security bug patterns to look for, 
both these tools will allow for greater accuracy of targeting security bugs.

The security badge represents the number of security bugs reported by SpotBugs \cite{spotbugs} with the FindSecBugs \cite{findsecbugs} plugin.
We filter for only security bug patterns and configure SpotBugs to only classify bugs on the highest confidence setting
with maximum effort toggled on to increase precision. To the user, this badge helps to detect
for any known security vulnerabilities before using the library. 

\subsubsection{Implementation}

\begin{figure}[!htb]
    \centerline{
        \includegraphics[width=6cm,height=6cm,keepaspectratio=true]{findsecbugsbadge}
    }
    \caption{
        Example of Security Badge
    }
    \label{findsecbugsbadge}
\end{figure}

First, we have a text file that holds open source Java library links hosted on GitHub \cite{github}.
A shell script clones and compiles them respectively under Gradle \cite{gradle} or Maven \cite{maven}.
After compilation, a script runs SpotBugs and FindSecBugs per library under our defined configured settings
and stores the respective result
into the database. The client can hit the security script endpoint to retrieve the saved results which
then can be placed into Shields.io \cite{shields} to output the security badge with an example shown in figure \ref{findsecbugsbadge}. 


\subsection{Last Discussed on Stack Overflow}
\subsubsection{Definition}
We take Mora's et. al \cite{metrics} Last Discussed on Stack Overflow metric and transform it 
into a badge. Based on the metric, the badge represents the latest date of a question posted for a specific
library on the Stack Overflow website \cite{stackoverflow}.
To the user, this badge will display if there is any recent activity using the library and the community
involvement behind it.  

\subsubsection{Implementation}

\begin{figure}[!htb]
    \centerline{
        \includegraphics[width=10cm,height=10cm,keepaspectratio=true]{lastdiscussedbadge}
    }
    \caption{
        Example of Last Discussed on Stack Overflow Badge
    }
    \label{lastdiscussed}
\end{figure}

Borrowing the implementation technique from \cite{metrics}, we use the StackExchange API \cite{stackexchangeapi}
to search up the library's name under tag search and extract the most popular tag.
With the tag, we make a GET request to the API to search for the most recent question containing the tag
and extract the date. An example of the badge is shown in figure \ref{lastdiscussed}.

\subsection{Issue Response Time}
\subsubsection{Definition}
Using Mora's et. al \cite{metrics} Issue Response Time definition, this badge illustrates the average
amount of days needed to get the first reply to an issue. The badge is calculated via the response time
formula described in \cite{metrics}. It also indicates if there are any changes to the average over time 
as shown by the rightmost symbol in \ref{issueresponse}. 
This badge will allow users to check if there is active maintainence of the library and if there is community support
through the issue discussions. 

\subsubsection{Implementation}

\begin{figure}[!htb]
    \centerline{
        \includegraphics[width=10cm,height=10cm,keepaspectratio=true]{issueresponsebadge}
    }
    \caption{
        Example of Issue Response Time Badge
    }
    \label{issueresponse}
\end{figure}



TODO WRITE UP METHODOLOGY
We also discard deleted accounts and do not take them into account in our average.
An example of the badge is shown in figure \ref{issueresponse}.


\subsection{Contributor Pull Request Merge Rate}
\subsubsection{Definition}
Kononenko et. al \cite{shopifyarticle} highlights that, \textit{PRs submitted by the owners are approved more
quicklier compared to those made by external devs}. This can hinder open source development
for developers who wish to partake in the project but are not core maintainers or closely associated with the project.
In addition, there is the lack of assessment defined badges \cite{githubbadges} related to pull requests (PRs)
as of December 2018.
The existing PR badges on Shields.io \cite{shields} display readily available PR information that can be easily found on a 
software repository. Taking both resources \cite{shields, shopifyarticle} into account, this has inspired
the development of the Outside Contributor Pull Request Merge Rate assessment badge. 


The badge represents the percentage of PRs that have been merged into the repository authored by outside contributors.
We define outside contributors using the survey design outlined by Trockman et. al \cite{githubbadges} if they have 
less than 10\% of commits made to a repository.
This badge helps users to select libraries that are friendly and open to open source contributions,
and easily evaluate the merge rate not accessible beforehand.
With a higher merge rate, it will provide opportunities for developers to extend features of a library with 
higher confidence that their PR will be integrated into the repository.

\subsubsection{Implementation}

\begin{figure}[!htb]
    \centerline{
        \includegraphics[width=10cm,height=10cm,keepaspectratio=true]{prbadge}
    }
    \caption{
        Example of Outside Contributor PR Merge Rate Badge 
    }
    \label{prbadge}
\end{figure}

We have a seperate script endpoint that uses the GitHub \cite{github} API to get all the users
that have made commits to the library. We classify users as outside contributors or not using
the technique described by Trockman et. al \cite{githubbadges} and store them as a JSON object in the database.
Completing this task will allow us to use another endpoint to calculate the PR metric.
With the GitHub \cite{github} API, we calculate
the average as follows: total number of merged PRs divided by total number of all PRs where 
PRs of all statuses are associated with outside contributors. An example response is shown in figure \ref{prbadge}.


\subsection{Release Frequency}
\subsubsection{Definition}
TODO

\subsubsection{Implementation}

\begin{figure}[!htb]
    \centerline{
        \includegraphics[width=10cm,height=10cm,keepaspectratio=true]{releasebadge}
    }
    \caption{
        Example of Release Frequency Badge
    }
    \label{releasebadge}
\end{figure}

We use the GitHub \cite{github} API to get all the releases of the library and follow the algorithm
steps shown in \cite{metrics} to calculate the release frequency metric.
The script makes another request with a list of commits associated with releases as input
and outputs a new list of dates which then can help determine the final average using the suggested
formula.




\section{Evaluation}

\subsection{Open Source Evaluation}
% Shields.io Open source contributions
% Choose SO badge b/c it has the least amount of data to process and gives us a good
% start to see if prototyping thsi works 
\subsection{Future Work Evaluation}
% If I had 4 more months, how would I evaluate this work.
% Have badges hosted on Shields and figure how how many people are using them for popularity?
% Survey insights to why they are suing those specific badges

\section{Threats to Validity and Limitations}
\subsection{Validity}
\subsubsection{Classification of Outside Contributors}
% <10% comntibnutor rate doesn't capture misclassifies some to be contribtuor when in reality are maintainers
\subsubsection{Security Bug Classification}

\subsection{Limitations}
\subsubsection{Request Timeouts}
All metrics are susupetible to this due to GitHit image restrictions  3 seconds or less. 
\subsubsection{Local Scripts}
% Database caching
% Security only works with Java Libraries and requires seperate update 
% SCripts can only be used locally and need to rerun ngrok everytime


\section{Conclusion and Future Work}
% Future work
% Host Scripts openly using something like heroku or now.js
% Improve Error handling to make sure client knows any bad http codes
% make front end application to display these metrics to get detailed info, visualizations of metrics
% being used where and number of people suing them in line graph


\newpage
\bibliographystyle{IEEEtran}
\bibliography{bibliography}
\end{document}